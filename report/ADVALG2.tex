\documentclass[9pt]{article}
\usepackage[left=12mm,top=0.5in,bottom=0.5in]{geometry}

\title{3805ICT - Assignment 2}
\author{Damon Murdoch (s2970548)}

\begin{document}
\maketitle

\section{Introduction}

The molecular model problem is based around the concept of 'N' atoms connected by rigid bonds of unit length constrained in two dimensional space. The potential energy of such a system, 'V' is given by the scaled, pairwise addition of Lennard-Jones potentials. The purpose of this investigation is to develop an algorithm for minimising the value of 'V' for a molecular model for any given 'N' number of atoms. The algorithm will be implemented using the limited memory Broyden-Fletcher-Goldfarb-Shanno (L-BFGS) algorithm to minimise the energy in the system iteratively, using the gradient of the change in energy for each atom at each step to minimise the energy cost of the system. While this problem operates on an extremely simple representation of an atom in two dimensional space, the investigation is useful for developing molecular structure optimisation algorithms as the minimum energy for the system can be determined for any dimension without undue complexity. The following report will provide a detailed description of the problem, the algorithm implemented to solve the problem and the resulting test data provided by the investigation.

\section{Literature Review}

A molecule is defined as "the simplest unit of a chemical substance, usually a group of two or more atoms "(Dictionary, 2018). an atom is defined as "the smallest unit of any chemical element, consisting of a positive nucleus surrounded by negative electrons"(Dictionary, 2018). Understanding the stable configurations of a molecule is important because it enables us to understand its properties and behavior with respect to its structure. When a molecule is constructed using computational chemistry software it may not be given a stable initial state. This means that if the same molecule were to be constructed in reality, it may be unstable and behave unexpectedly. As a result of this, energy minimisation also known as geometry optimisation is performed to find a stable  state for the molecule.
\\
\\
Energy Minimisation is a numerical procedure which is used to find the minimum potential energy of a molecular system starting from a higher energy initial structure. During this process, the geometry of the molecule is adjusted iteratively such that the overall potential energy of the system is reduced. The search is terminated after a set number of iterations are performed or an accepted global minimum point is reached. CASTEP is a quantem mechanics based program designed for solid-state materials science, and can be utilised for geometry optimisation using BFGS and damped molecular dynamics. The BFGS minimiser is used primarily as it has the ability to perform cell optimisation, including optimisation at fixed external stress levels. This implementation involves a Hessian model in the mixed space of internal and cell degrees of freedom, in order to ensure both lattice parameters and atomic coordinates can be optimised. While left off by default, constraints can be applied to the model such as the fixing of atom positions, or fractional coordinates of the atoms.
\\
\\
Crossovers are a particularly significant issue which can arise during the run-time of a solving algorithm, and involves the crossing over of two different line segments in a bond constrained model. Crossovers are difficult to detect in a way which is not computationally expensive, and can often result in a state becoming close to but unable to reach the optimal system energy. They are particularly difficult to avoid in population based search implementations, as they rely on having a large pool of solution states which are combined at every iteration and presere the environments closest to the desired solution at each iteration. Due to the nature of these algorithms involving random combinations of atom positions, it is extremely likely for these combinations to introduce crossovers. Crossovers can be fixed by recursively straightening out the angle to the next atom in the chain associated with the offending atoms until no more crossovers occur, however this is expensive and it is possible to involve almost completely resetting a solution state to its starting positions. In general, it is better to avoid causing crossovers in the first place. 
\\
\\
This review's purpose is to summarise existing research into molecular structure optimisation, as well as define key words and algorithms which are frequently used with respect to this problem and associated concepts. There are few studies which have been performed on the two dimensional molecular optimisation problem as it is effectively just a simplication of the three dimensional molecular optimisation problem, which is far more useful for providing models which can be used in real experiments. However, the algorithms which can be used to solve the three dimensional problem which have been discussed can be applied in a similar fashion to the two dimensional problem.

\section{Algorithm Description}

For this problem, there are two procedures which will be implemented and utilised cooperatively in order to find a solution. The first algorithm is an over-arching algorithm for making general changes to the overall solution state in order to escape local minima, and and the second for making precise adjustments that lead straight to a local minimum  from the current atom configuration. For the general case algorithm, heuristic based local search algorithms are utilised. For this investigation, we will primarily be looking at population based genetic search algorithms and the simulated annealing method of hill climbing search. The second algorithm utilised 

\subsection{General Case Algorithms}

The following section of the report will provide descriptions of the general case algorithms utilised in this study.

\subsubsection{Genetic Algorithm}

\paragraph{Algorithm Description}

A genetic algorithm is a population-based local search algorithm which is built following the principles of genetic diversity and 'survival of the fittest' in biological systems. It works by generating a set number 'N' of random initial configurations, performs random crossovers of each state and preserves a set number of 'best' solutions 'P' sorted by cost. Due to the very nature of genetic algorithms being crossover based, these algorithms are extremely susceptible to 'flooding', or filling the entire preserved population set with solutions with the same values. In order to prevent this, 'mutations' or random modifications to single or multiple values of a particular solution are utilised however it is often difficult to find the right balance between flooding and 'pollution', or filling the solution with poor board states that slow the search down heavily. This is a major issue for implementations which rely exclusively on genetic algorithms, however for this problem the utilisation of a secondary algorithm provides a method of escaping population induced local minima.

\subsubsection{Simulated Annealing Algorithm}

\paragraph{Algorithm Description}

Simulated annealing is a probibilistic search algorithm for approximating the global optimum of a given function. It is based off the process of annealing in metallurgy and materials science, and works by allowing the algorithm to accept worse solutions than the current state moves early in the search process, but slowly gets stricter on the solutions it can accept as it approaches lower temperatures. Temperature cooling can be simply based upon the number of iterations, or on more complicated cooling functions based upon the 'goodness' of the current solution state.  Simulated annealing has the strengths of requiring less memory than genetic based search algorithms, as you only need to store enough data for one system however it can take longer to reach a solution due to having a lower search space due to running on a single solution. 

\subsection{Secondary Algorithms}

The following section of the report will provide descriptions of the secondary algorithms utilised in this study.

\subsubsection{Broyden-Fletcher-Goldfarb-Shanno (BFGS) Algorithm}

\paragraph{Algorithm Description}

The BFGS algorithm for numerical optimisation is used to minimise a function f(x), where 'x' is a vector in $R^{n}$, and 'f' is a differentiable scalar function. There are no constraints enforced on the value of 'x' by the algorithm. It operates by taking an initial estimate of the optimal solution $x_{0}$ and proceeds to find a better estimate of the solution each iteration. The search direction $p_{k}$ at stage 'k' is given by the analogue of the solution of the Newton equation
\\ 
\\
$B_{k}$$P_{k}$ = -$\nabla$f($x_{k}$)
\\
\\
Where $B_{k}$ is the iterative approximation of the Hessian matrix of the problem and $\nabla$f($x_{k}$) is the gradient for the function evaluated at the position $x_{k}$. With respect to the two dimensional molecular optimisation problem, the gradient of the function at each point is the overall impact rotating the angle of the atom $a_{k}$ with respect to the atom $a_{k-1}$ on the total energy of the system. An alternative method for energy minimisation presented by CASTEP is damped molecular dynamics, which involves only internal coordinates for each atom and uses critical damping to deal with the ground state. This can be implemented by using either a single damping coefficient for all degrees of freedom for coupled nodes or by using coefficients for each degree of freedom for independent nodes. However, for this problem it can be observed that BFGS algorithms are much more commonly utilised than gradient descent methods such as stochastic gradient descent as they generally require fewer iterations to reach an acceptable solution state.

\section{Results Discussion}

\section{Bibliography}

Dictionary, a. (2018). atom Meaning in the Cambridge English Dictionary. [online] Dictionary.cambridge.org. 
\\Available at: https://dictionary.cambridge.org/dictionary/english/atom [Accessed 24 Jul. 2018]
\\
\\
Dictionary, m. (2018). molecule Meaning in the Cambridge English Dictionary. [online] Dictionary.cambridge.org.
\\Available at: https://dictionary.cambridge.org/dictionary/english/molecule [Accessed 24 Jul. 2018]
\\
\\
shodor.org. (2018). Background Reading for Geometry Optimizations. [online] 
\\Available at: https://www.shodor.org/chemviz/optimization/students/background.html [Accessed 24 Jul. 2018]
\\
\\
Structure.usc.edu. (2018). Minimization and Molecular Dynamics. [online] 
\\Available at: http://structure.usc.edu/mmtk/MMTK\_4.html [Accessed 24 Jul. 2018]
\\
\\
Jean, M. (2018). Energy Minimization Methods. [online]  About.illinoisstate.edu.
\\Available at: https://about.illinoisstate.edu/standard/Documents/CHE\%20380.37/Handouts/380.37emin.pdf [Accessed 25 Jul. 2018].
\\
\\
Clark, S. (2018). Geometry Optimization. [online] Cmt.dur.ac.uk. 
\\Available at: http://cmt.dur.ac.uk/sjc/thesis\_dlc/node37.html [Accessed 26 Jul. 2018].
\\
\\
Ul-Haq, Z. (2018). Introduction to Geometry Optimization. [online] Th.fhi-berlin.mpg.de. 
\\Available at: https://th.fhi-berlin.mpg.de/sitesub/meetings/dft-workshop-2016/uploads/Meeting/May\_6\_Qasmi.pdf [Accessed 26 Jul. 2018].
\\
\\
Spindynamics.org. (2018). molecular geometry optimization. [online]
\\Available at: http://spindynamics.org/documents/cqc\_lecture\_6.pdf [Accessed 26 Jul. 2018].
\\
\\
Helgaker, T. (2009). Geometry optimization. [online] Folk.uio.no. 
\\Available at: http://folk.uio.no/helgaker/talks/ESQC09\_Optimization.pdf [Accessed 26 Jul. 2018].
\\
\\
Openmopac.net. (2018). The BFGS function optimizer. [online] 
\\Available at: http://openmopac.net/manual/BFGS\_optimizer.html [Accessed 26 Jul. 2018].
\\
\\
Tcm.phy.cam.ac.uk. (2018). CASTEP Geometry optimization. [online] 
\\Available at: https://www.tcm.phy.cam.ac.uk/castep/documentation/WebHelp/content/modules/castep/thcastepgeomopt.htm [Accessed 26 Jul. 2018].
\\
\\
Cpmd.org. (2018). Geometry Optimization. [online] 
\\Available at: http://www.cpmd.org:81/manual/node80.html [Accessed 26 Jul. 2018].
\\
\\
General methods for geometry and wave function optimization
\\Thomas H. Fischer and Jan Almlof
\\The Journal of Physical Chemistry 1992 96 (24), 9768-9774
\\DOI: 10.1021/j100203a036
\\
\\
Tcm.phy.cam.ac.uk. (2018). CASTEP. [online] 
\\Available at: https://www.tcm.phy.cam.ac.uk/castep/documentation/WebHelp/content/modules/castep/abtcastep.htm [Accessed 26 Jul. 2018].
\\
\\
Clark, S. J.; Segall, M. D.; Pickard, C. J.; Hasnip, P. J.; Probert, M. J.; Refson, K.; Payne, M. C. 
\\"First principles methods using CASTEP", Zeitschrift fuer Kristallographie, 220 (5-6), 567-570 (2005).
\\
\\
\end{document}